Now that the requirements necessary to build the system have been formulated, potential solutions for each aspect of the system can now be analysed before a final solution can be chosen.

\section{Pipeline Design Analysis}

Explore possible options for different sections of the system, such as:
    \begin{itemize}
        \item types of features to extract (why static colour features and not object/motion features) - histograms are very popular with videos, calculations are easier, implementation is easier, results are as efficient
        \item types of learning models (histogram matching, BoW-approacg VS Neural Network)
    \end{itemize}

\subsection{Offline Feature Extraction Phase}

todo

\subsection{Online Retrieval Phase}

todo

\subsection{Database Pre-Processing Phase}

todo

\section{General Project Design}

decision on programming language, on database videos and interface choice

\subsection{Programming Language}

\begin{itemize}
	\item python VS MATLAB vs other languages
	\item rich array of libraries: OpenCV, NumPy, SciPy, MatplotLib
	\item ease of installing third-party libraries with PIP
	\item powerful IDEs (PyCharm)
	\item ease of generating testing suites
     \item familiarity with
\end{itemize}

\subsection{Interface}

interface choice (why CLI VS GUI?): time constraints and no users, goal of this project is to research efficient results, not create a commercial product

\subsection{Database videos}

types of videos in databases (why short videos VS movies, cartoons/stop-motion pictures)

\section{Chosen Solution}

include a high-level diagram of system excluding detail (all DB videos in system, single query video in system, matching video output)

\section{Development Plan}

plan with time constraints in mind

\section{Summary}

summarise section
