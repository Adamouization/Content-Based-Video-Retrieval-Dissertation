The aim of this chapter is to review how the system is tested in order to assess its quality as a whole. The three phases of the pipeline are analysed separately, with an emphasis on the online retrieval phase that produces the output of the system. The type of database videos and query videos used to test the system is clarified before diving into the evaluation of the results, starting with the online retrieval phase, followed by the offline feature extraction phase and ending with the database pre-processing phase. These phases are analysed with the help of software evaluation tools, along with figures and graphs utilised for visual representations of the results. Finally, an online experiment is conducted to establish ground truth matching and compared to the system's own result.

%%%%%%%%%%%%%%%%%%%%%%%%%%%%%%%%%%%%%%%%%%%%%%%%%%%%%%%%%%%%%%%%%%%%
%%%%%%%%%%%%%%%%%%%%%%%%%%%%%%%%%%%%%%%%%%%%%%%%%%%%%%%%%%%%%%%%%%%%
%%%%%%%%%%%%%%%%%%%%%%%%%%%%%%%%%%%%%%%%%%%%%%%%%%%%%%%%%%%%%%%%%%%%

\section{Testing Data}

\subsection{Database Videos}

The system is tested using a database of 50 different videos, ranging from 7 to 14 seconds, as per requirements F25-F26. The database of videos is populated from scratch using free stock footage from \textit{Pexels Video}\footnote{Pexels Video: \url{https://www.pexels.com/videos/}}. The videos used to build the database are chosen from a rich library of videos encapsulating many different colours (e.g. bright, dark, warm, cold, colourful, etc.) and movements (e.g. still, motion, shaky, blurry, timelapses, etc.) to ensure diversification in the dataset. The database of videos used to test the system can be found in the \textit{``footage''} directory in the provided code.\\

Employing existing databases of videos for CBVR tasks such as the TRECVID\footnote{Text REtrieval Conference Video Retrieval Evaluation} dataset \cite{2018trecvidawad}, which is the best dataset for CBVR-oriented tasks, would have been ideal to measure the project's performance and to compare it with existing solutions for this task. However, these datasets are not publicly available and hard to obtain. For example, most of the referenced proceedings in the Bibliography that present solutions to CBVR tasks were conducted for the TRECVID conferences, meaning that the datasets used to test the systems were provided to the participants for testing and evaluating results. However, the NIST\footnote{National Institute of Standards and Technology, organiser of the annual TRECVID conference} does not provide these databases for external research. Other databases of videos exist, but target other computer vision tasks such as facial recognition or image retrieval rather than CBVR. Therefore, the previously mentioned custom database of 50 copyright-free videos is used to test this system.

%%%%%%%%%%%%%%%%%%%%%%%%%%%%%%%%%%%%%%%%%%%%%%%%%%%%%%%%%%%%%%%%%%%%

\subsection{Query Videos}

Various query videos are recorded to test the online retrieval phase of the system. These queries are mobile recordings of one of the database videos. Different types of queries listed in Section \ref{sec:design-query-video-processing} are used to test the limits of the system, including down-scaled (recording at a distance from the screen) and skewed queries (recording at an angle) coupled with minor camera movement due to shaking hands. These conditions ensure the realism of the queries if the system were to be developed as a mobile application.

%%%%%%%%%%%%%%%%%%%%%%%%%%%%%%%%%%%%%%%%%%%%%%%%%%%%%%%%%%%%%%%%%%%%
%%%%%%%%%%%%%%%%%%%%%%%%%%%%%%%%%%%%%%%%%%%%%%%%%%%%%%%%%%%%%%%%%%%%
%%%%%%%%%%%%%%%%%%%%%%%%%%%%%%%%%%%%%%%%%%%%%%%%%%%%%%%%%%%%%%%%%%%%

\section{Online Retrieval Phase Results Evaluation}

The results are evaluated by counting the number of true positives and false positives for each video query used to test the system. A true positive corresponds to an outcome where the system matches the query to the correct database video, while a false positive is an outcome where the system matches the query incorrect to an incorrect database videos. The number of true positives and false positives occurrences are counted for each histogram model and distance metric used per query. As detailed in the previous chapter at the end of Section \ref{sec:implementation-distance-measurements}, these are used to plot the probabilities of the most likely database video to match the query in the form of a percentage \%. Additionally, other measurements are made such as the runtime of the system, its scalability, and the comparison of the final results under different conditions.

%%%%%%%%%%%%%%%%%%%%%%%%%%%%%%%%%%%%%%%%%%%%%%%%%%%%%%%%%%%%%%%%%%%%

\subsection{Video Matching Performance}

\subsubsection{Accuracy Measurements}

The system is first tested with down-scaled queries recorded at a straight angle to the screen. Shaky camera movement can be seen on the query videos, along with a considerable section of frame area not covering the actual video playing due to the distance to the screen. Despite these challenges, the queries still yield correct matches with high accuracy exceeding 50\% of true positives, with some excellent results that exceed 90\%. Figure \ref{fig:evaluation-downscaled-queries} depicts two examples of queries that incorporate the specified challenges, with the first query retrieving the correct video with 93.18\% accuracy, and the second one with 56.82\%. The important detail to notice in the left-hand side histograms is how the probability of the second closest match is much smaller than the probability of the closest match. Additionally, few alternative videos are marked as potential matches, with the first query only displaying two potential matches and the second query four potential matches. % recording1 and recording2

\begin{figure}[h] 
\centerline{\includegraphics[width=\textwidth]{figures/evaluation/downscaled-queries.png}}
\caption{\label{fig:evaluation-downscaled-queries}Down-scaled Queries.}
\end{figure}

Next, skewed queries recorded at an angle from the screen playing the database video are tested. These tested queries also include all of the challenges mentioned in the first test, such as down-scaling (recording at a distance) and unstable footage. Despite the increased challenge presented by the skewed query, the accuracy of the matches still exceed 50\% and reach as high as 75\% despite poor quality queries. Figure \ref{fig:evaluation-skewed-queries} portrays examples of skewed queries. The first query filmed from the left side of the screen is identified as the correct match with 56.82\% accuracy, while the second query filmed from the right wide of the screen is identified as the correct match with 75\% accuracy. Observing at the left-hand side histograms in Figure \ref{fig:evaluation-skewed-queries}, it can be noticed that skewed queries cause the system to produce more potential video matches. Indeed, the first query produces six potential videos matches, which are quite different from the mainly-green \textit{``butterfly''} query, e.g. the main colours in the \textit{``autumn''} and \textit{``dunes''} are not green. This contrasts with the non-skewed queries that produce less potential matches. Nevertheless, the probability of the closest match remains greater than the probability of the second closest match despite the skewed query. % recording3 and recording8

\begin{figure}[h] 
\centerline{\includegraphics[width=\textwidth]{figures/evaluation/skewed-queries.png}}
\caption{\label{fig:evaluation-skewed-queries}Skewed Queries.}
\end{figure}

Although, the results demonstrated in the first two tests using down-scaled and skewed queries portrayed positive results with more than 50\% true positives, some queries result in a poorer accuracy, reaching a minimum of 45.5\% for a correct match, as shown in Figure \ref{fig:evaluation-poor-accuracy-query}. This low accuracy is partly caused by a combination of factors that are further explored in the next paragraph. % recording5

\begin{figure}[h] 
\centerline{\includegraphics[width=\textwidth]{figures/evaluation/poor-accuracy-query.png}}
\caption{\label{fig:evaluation-poor-accuracy-query}Poor accuracy Query.}
\end{figure}

%%%%%%%%%%%%%%%

\subsubsection{External Factors Observations}
\label{sec:evaluation-observations-online-phase}

External factors directly affecting the quality of the query can drastically influence the accuracy of the system, and even produce wrong matches. These factors are causes by the conditions of the environment rather than by the actual recording's quality.\\

The first noticeable environmental factor that causes the system to produce low-accuracy results and possibly wrong matches is the lighting in the room. While testing the system with different queries, the system had trouble coping with queries recorded at night-time in low luminosity environments. Indeed, during the filming of the queries at night-time, the lamps in the room produced warm white light with colour temperatures ranging between 2000K and 3000K\footnote{What is colour temperature? \url{http://www.westinghouselighting.com/color-temperature.aspx}}, which corresponds to orange/yellow light. These caused an alteration of the overall colour of the recorded screen, leading to the histograms to shift towards yellow/orange colours. This shift provoked the query's average histograms to be very disparate from the original database video's histograms, ultimately causing the system to generate the wrong input. Two cases of the consequences of this type of environment are illustrated in Figure \ref{fig:evaluation-yellow-light-queries}. % recording4 and mismatch1

\begin{figure}[h] 
\centerline{\includegraphics[width=\textwidth]{figures/evaluation/yellow-light-queries.png}}
\caption{\label{fig:evaluation-yellow-light-queries}Yellow light Queries.}
\end{figure}

The query from the first case (top of Figure \ref{fig:evaluation-yellow-light-queries}) is matched to the correct database video, but with a very low accuracy of 45.45\%. Furthermore, the system found five other potential matches, most with shades of yellow/orange colours (e.g. \textit{``dunes''}, \textit{``bird-walking''} and \textit{``offroad-car''}). The second query used from Figure \ref{fig:evaluation-yellow-light-queries} does not even find the correct match, pairing the \textit{``butterfly''} query to the \textit{``bird-walking''} video. The system does not even list the correct video as a potential match and moreover places the closest and second closest matches near each other (within 5\%). After watching the second query of the butterfly in more detail, it can be noticed that the dark environment is source to another factor. Due to the poorly lighted room, the luminosity of the screen stands out more, which can in turn shift the intensity of the pixels in the histogram towards the right (toward a value of 255). This contributes to the poor accuracy of the system in such conditions.\\

The second noticeable external factor is the presence of light reflections on the screen displaying the database videos. This factor, which is already probed in Section \ref{sec:design-query-video-processing} Figure \ref{fig:difference-query-video-issues}, depends on the positioning of the light source and the camera recording the query. Based on these, light glares might appear on the screen, greatly affecting the accuracy of the system as new colours appear on the screen, causing the histograms' pixel intensities to converge towards that new colour. Two cases reproducing the consequences of the light glares on the screen are represented in Figure \ref{fig:evaluation-light-reflection-queries}. % recording7 and mismatch2

\begin{figure}[h] 
\centerline{\includegraphics[width=\textwidth]{figures/evaluation/light-reflection-queries.png}}
\caption{\label{fig:evaluation-light-reflection-queries}Light Reflection Queries.}
\end{figure}

The first query of \textit{``people-dancing''} is correctly matched to its database video regardless of the light glare on the right-hand side of the screen (highlighted in red), but again with a low accuracy of 45,45\%. The system detects potential matches in videos that are quite distant to the query video, such as \textit{``airplane-window''} or \textit{``flower''} which both have shades of blue similar to the glare. The second query of \textit{``spaghetti''} is matched to the wrong video, and is not even listed as a potential match despite its unique combination of yellow and pink colours because of the larger light glare in the bottom right of the screen (highlighted in red).\\

This section depicted the worst case performances of the system under difficult conditions, all caused by different lighting factors such as the colour of the light or the reflection of light on the screen playing the database videos. The next section critiques the combination of histogram models and distance metrics in more detail.

%%%%%%%%%%%%%%%%%%%%%%%%%%%%%%%%%%%%%%%%%%%%%%%%%%%%%%%%%%%%%%%%%%%%

\subsection{Histogram Models and Distance Metrics Analysis}

analyse results of RGB, grey scale and HSV histograms with the 6 different histogram matching methods. which work best? in which scenarios? (use the 10 previous results to show this)

removing KL divergence metric greatly improved accuracy. (using less metric to use the ones for the job). KL divergence NEVER found correct match, not meant to be used a distance metric (show example of removing KL div and alternate chi square with past examples and new one)

improved cloudy-sky,
improved butterfly skewed

histograms allow for any video quality as long as it is acceptable (quicker processingm same results)

%%%%%%%%%%%%%%%%%%%%%%%%%%%%%%%%%%%%%%%%%%%%%%%%%%%%%%%%%%%%%%%%%%%%

\subsection{Runtime Measurements}

The runtime of each of the ten query videos used to test the system in the previous sections is plotted in a graph which can be found in Figure \ref{fig:evaluation-runtime_plot}. A few points can be made based on this graph. First, the runtime never exceeds 10s, thus fulfilling requirement F13. Second, the difference between the 1080p query videos and lower-quality 720p query videos' runtime is almost halved, with the runtime of 720p queries averaging 4.89 seconds and the runtime of 1080p queries averaging 8.91 seconds. However, the accuracy of system when processing the 720p queries is as high as the accuracy of the 1080p queries. Indeed, the \textit{``cloudy-sky''} query yields an accuracy of 93.18\%, while the accuracy of the \textit{``jellyfish''} query yield 75\% true positives. This crucial measurement reveals that using higher-quality does not necessarily translate to better results, meaning that any decent mobile device camera can be used with the system to generate descriptive histograms.

\begin{figure}[h] 
\centerline{\includegraphics[width=\textwidth]{figures/evaluation/runtime_plot.png}}
\caption{\label{fig:evaluation-runtime_plot}Runtime plot.}
\end{figure}

%%%%%%%%%%%%%%%%%%%%%%%%%%%%%%%%%%%%%%%%%%%%%%%%%%%%%%%%%%%%%%%%%%%%
%%%%%%%%%%%%%%%%%%%%%%%%%%%%%%%%%%%%%%%%%%%%%%%%%%%%%%%%%%%%%%%%%%%%
%%%%%%%%%%%%%%%%%%%%%%%%%%%%%%%%%%%%%%%%%%%%%%%%%%%%%%%%%%%%%%%%%%%%

\section{Offline Feature Extraction Phase Scalability}

Evaluating the offline colour-based feature extraction phase is more tricky than evaluating the online retrieval phase as the output is always the same between different runs: three averaged histograms for each database video. This phase is nevertheless essential as it decides how large the database of videos can be. Therefore, a quantitative measure can be used to determine the scalability of this phase and how large the database can be.\\

The measure used to estimate the scalability of the system is the runtime of the phase. The mean runtime for generating the three averaged histograms for each video in a database of 50 videos is 163 seconds. With the way the features are extracted and the feature vectors are generate, the growth of the system is linear as it is equal to $163/50=3.26$ seconds per video. Adding a 51st video to the database will increase the runtime by 3.26 seconds. To predict how large the database of videos can be, the runtime in hours is plotted in a graph starting with 50 videos and growing up to one million videos based on the previous linear calculations. The results can be found in Figure \ref{fig:evaluation-offline_phase_runtime_trendline}.

\begin{figure}[h]
\centerline{\includegraphics[width=\textwidth]{figures/evaluation/offline_phase_runtime_trendline.png}}
\caption{\label{fig:evaluation-offline_phase_runtime_trendline}Prediction of the offline colour-based feature extraction phase runtime for different database sizes. A trendline $y=0.021e^{0.7667x}$ can be fitted to the resulting series to determine the demand }
\end{figure}

These results clearly betray the demand of processing large databases of videos. Up until 1000 videos, the runtime of processing the database would be inferior to one hour. The runtime reaches 9 hours for reaching 10000 videos, which is still acceptable to calculate in a single execution. However, the runtime to process larger databases would need to be calculated in days rather than hours when the database reaches one million videos, with runtimes of $905.61/24=37.73$ days. This would be feasible with dedicated hardware running this phases through multiple batches and building the compact signature for each database video over time, but not for the current system.

%%%%%%%%%%%%%%%%%%%%%%%%%%%%%%%%%%%%%%%%%%%%%%%%%%%%%%%%%%%%%%%%%%%%
%%%%%%%%%%%%%%%%%%%%%%%%%%%%%%%%%%%%%%%%%%%%%%%%%%%%%%%%%%%%%%%%%%%%
%%%%%%%%%%%%%%%%%%%%%%%%%%%%%%%%%%%%%%%%%%%%%%%%%%%%%%%%%%%%%%%%%%%%

\section{Database Pre-Processing Test: Movie Segmentation}

The shot boundary detection algorithm is used on a feature-length movie. The movie used for the test is Inception\footnote{Inception IMDb page: \url{https://www.imdb.com/title/tt1375666/}}. The movie is composed of 213098 frames.

The shot boundary detection algorithm is used once with the KL Divergence and once with the Intersection metric..

\begin{itemize}
    \item KL Divergence: a global threshold of 10 using the KL Divergence detects 661 shot boundaries (0,31\%) in 149.7 minutes.
    \item Intersection: A global threshold of 7 using the Intersection metric detects 1730 shot boundaries (0.81\%).
\end{itemize}

\begin{figure}[h] 
\centerline{\includegraphics[width=1.15\textwidth]{figures/evaluation/inception_shot_boundary_detection_test.png}}
\caption{\label{fig:evaluation-inception_shot_boundary_detection_test}Result of the shot boundary detection algorithm on the Inception movie using the KL Divergence (left) and Intersection metric (right) between consecutive frames' RGB histograms.}
\end{figure}

%%%%%%%%%%%%%%%%%%%%%%%%%%%%%%%%%%%%%%%%%%%%%%%%%%%%%%%%%%%%%%%%%%%%
%%%%%%%%%%%%%%%%%%%%%%%%%%%%%%%%%%%%%%%%%%%%%%%%%%%%%%%%%%%%%%%%%%%%
%%%%%%%%%%%%%%%%%%%%%%%%%%%%%%%%%%%%%%%%%%%%%%%%%%%%%%%%%%%%%%%%%%%%

\section{Comparison With Ground Truth Experiment}

online classification experiment:
    \begin{itemize}
        \item Google Survey experiment results 
        \item Compare experiment results with the algorithm results
        \item use a plot to visualise experiment results
        \item link to appendix (Ethics Checklist Appendix \ref{ch:appendix-ethics-checklist}, Experiment Script Appendix \ref{ch:appendix-experiment-survey}, and Raw Results)
    \end{itemize}
    
mention ground truth (https://en.wikipedia.org/wiki/Ground\_truth) (https://www.techopedia.com/definition/32514/ground-truth)

%%%%%%%%%%%%%%%%%%%%%%%%%%%%%%%%%%%%%%%%%%%%%%%%%%%%%%%%%%%%%%%%%%%%
%%%%%%%%%%%%%%%%%%%%%%%%%%%%%%%%%%%%%%%%%%%%%%%%%%%%%%%%%%%%%%%%%%%%
%%%%%%%%%%%%%%%%%%%%%%%%%%%%%%%%%%%%%%%%%%%%%%%%%%%%%%%%%%%%%%%%%%%%

\section{Summary}

todo