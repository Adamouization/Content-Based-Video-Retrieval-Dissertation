% Please add the following required packages to your document preamble:
% \usepackage{longtable}
% Note: It may be necessary to compile the document several times to get a multi-page table to line up properly
\begin{longtable}[c]{|l|l|}
\hline
\multicolumn{1}{|c|}{\textbf{Text files}} & \multicolumn{1}{c|}{\textbf{Binary files}} \\ \hline
\endhead
%
\begin{tabular}[c]{@{}l@{}}Are human readable: text files \\ are easier to debug since \\ people can see what was \\ written to the file.\end{tabular} & \begin{tabular}[c]{@{}l@{}}Are not human readable: binary \\ files are a lot more difficult to \\ debug because people cannot \\ easily see what was written to \\ the file.\end{tabular} \\ \hline
\begin{tabular}[c]{@{}l@{}}When dealing with large data, \\ compression may be a factor. \\ For example, a 10-digit number \\ will take at least 10 bytes as text.\end{tabular} & \begin{tabular}[c]{@{}l@{}}Will store the same data in less \\ space. For example: a 10-digit \\ number will take as little as \\ four or two as binary.\end{tabular} \\ \hline
\begin{tabular}[c]{@{}l@{}}More restrictive than binary \\ files since they can only \\ contain textual data.\end{tabular} & \begin{tabular}[c]{@{}l@{}}Binary file formats are less \\ restrictive as they may include \\ multiple types of data in the \\ same file such as image, \\ video, and audio data.\end{tabular} \\ \hline
\begin{tabular}[c]{@{}l@{}}Less likely to become corrupted. \\ A small error in a text file may \\ simply show up once the file has \\ been opened.\end{tabular} & \begin{tabular}[c]{@{}l@{}}More likely to become corrupted. \\ A small error in a binary file may \\ make it unreadable.\end{tabular} \\ \hline
\begin{tabular}[c]{@{}l@{}}Many programs are capable of \\ reading and editing text files.\end{tabular} & \begin{tabular}[c]{@{}l@{}}Less programs are capable of reading \\ and editing text files.\end{tabular} \\ \hline
\caption{Table comparing the pros and cons for storing data in text files and binary files.}
\end{longtable}