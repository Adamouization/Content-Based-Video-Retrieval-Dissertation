% Please add the following required packages to your document preamble:
\begin{table}[h]
\centering
\begin{tabular}{@{}ll@{}}
\toprule
\multicolumn{1}{c}{\textbf{Text files}} & \multicolumn{1}{c}{\textbf{Binary files}} \\ \midrule
\multicolumn{1}{|l|}{Are human readable: text files are easier to debug since people can see what was written to the file.} & \multicolumn{1}{l|}{Are not human readable: binary files are a lot more difficult to debug because people cannot easily see what was written to the file.} \\ \midrule
\multicolumn{1}{|l|}{When dealing with large data, compression may be a factor. For example, 10-digits number will take at least 10 bytes as text.} & \multicolumn{1}{l|}{\begin{tabular}[c]{@{}l@{}}Will store the same data in less space. For example: 10-digits number will take as\\ little as four or two as binary.\end{tabular}} \\ \midrule
\multicolumn{1}{|l|}{More restrictive than binary files since they can only contain textual data.} & \multicolumn{1}{l|}{Binary file formats are less restrictive as they may include multiple types of data in the same file such as image, video, and audio data.} \\ \midrule
\multicolumn{1}{|l|}{Less likely to become corrupted. A small error in a text file may simply show up once the file has been opened.} & \multicolumn{1}{l|}{More likely to become corrupted. A small error in a binary file may make it unreadable.} \\ \midrule
Many programs are capable of reading and editing text files. & Less programs are capable of reading and editing text files. \\ \bottomrule
\end{tabular}
\end{table}