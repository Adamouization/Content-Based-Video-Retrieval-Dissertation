This chapter establishes the requirements of the system. These requirements were devised in response to the problems mentioned in Section \ref{sec:problem-description} of Chapter \ref{ch:chapter1}, along with the information provided by the Literature and Technology Survey in Chapter \ref{ch:chapter2}.\\

The following requirements are divided between non-functional and functional sections, which are further split into more precise sections. They are prioritised using the words \textit{must} (Priority 1), \textit{should} (Priority 2) and \textit{may} (Priority 3); where \textit{must} (Priority 1) indicates a crucial requirement to the system, \textit{should} (Priority 2) an optimal requirement and \textit{may} (Priority 3) a beneficial requirement.
%In the following requirements, the term ``\textit{query video}'' or ``\textit{recorded video}'' corresponds to the video that is used as input to be pattern matched to one of the videos in the database. It may either be directly recorded using a mobile device camera or it may be pre-recorded and passed as input.

\section{Non-Functional Requirements}

\begin{enumerate}
    \item The interface \textbf{should} be easy to read, using large fonts, with an emphasis on text used to discern the main results from .
    \item The system \textbf{may} not crash or exit unexpectedly with no error message.
\end{enumerate}

\section{Functional Requirements}

\subsection{System Requirements}

\subsubsection{Offline Feature Extraction Phase}

These requirements concern the offline feature extraction phase\footnote{The learning phase in a classifier ingesting static unsupervised data.} of the system, where static database videos are ingested by the system and features are extracted from each video and stored in files for the next phase.

\begin{enumerate}
    \item \textbf{The system \underline{must} successfully run this phase a single time and be able to run multiple times in the next phase}.\\
    The system \textbf{must} re-ingest the data whenever a change is carried out in the database e.g. a video has been added or a video has been removed.\\
    Priority 1.
    
    \item \textbf{The system \underline{must} store the extracted features it used for learning in files for future re-use in the online retrieval phase.}\\
    Files can be saved either in ``\textit{.txt}'' file format to be human-readable, or in other file formats such as ``\textit{.dat}'' files.
    
    \item \textbf{The files containing the training data \underline{must} be able to be quickly loaded into memory during the matching phase.}\\
    To minimise additional time spent on I/O operations, the files should be loaded quickly using efficient functions designed for this type of operation.
    
    % \item \textbf{The system \underline{should} extract static features into histograms.}\\
    % It should use either RGB histograms, grey scale histograms, HSV histograms, or a combination of multiple of the aforementioned histograms.
    
    \item \textbf{The system's interface \underline{}{should} specify that it is ingesting and extracting features from the database videos.}\\
    It can do so by either displaying a loading spinner or by iteratively printing text specifying which video is being currently processed.
\end{enumerate}

\subsubsection{Online Retrieval}

These requirements concern the retrieval phase\footnote{The matching phase of a classifier where the extracted features of a query video are compared to each database video's extracted feature.} of the system, where the same extracted features from the previous phase are extracted from the query video and compared to the stored features from each database video by calculating the distance between between these features.

\begin{enumerate}
    \item \textbf{The system \underline{must} accept a pre-recorded video as input.}\\
    This pre-recorded video will serve as the query video.
    
    \item \textbf{The system \underline{must} crop the recorded video to select a ``region of interest''.}\\
    It \textbf{must} ignore the pixels outside of the selected area of the clip.
    
    \item \textbf{The system \underline{must} display the results once the classification has been completed.}\\
    The retrieved video \textbf{must} be emphasised over the other videos to clearly point out that the result has been found. Additional results such as individual distances between the query and each video can be displayed as well e.g. tables.
    
    \item \textbf{The system \underline{should} stabilise the query video before extracting its features.}\\
    The stabilised video will diminish the effects of camera movement and increase the similarities between the video's extracted features and the database videos' features.
    
    \item \textbf{The system \underline{may} be able to directly record a query video.}\\
    Processing the query video while it is being recorded would drastically improve the system's runtime in this phase rather than waiting for the entire video to be recorded and only processing it after the recording is finished
    
    % \item The system \textbf{should} be able to find a match to the recorded video in less than 20 seconds.
    
    % \item The system \textbf{must} be able to match the recorded video with one of the videos in the database with at least 95\% accuracy.
\end{enumerate}

\subsubsection{Database Pre-processing}

These requirements concern the database pre-processing phase of the system. This phase would be used before the two previously mentioned phases, and concerns the operations to carry out the the database videos to improve the offline feature extraction and online retrieval phases.

\begin{enumerate}
    \item \textbf{Long videos e.g. feature-length movies, \underline{must} be processed for shot boundary detection to segment them into scenes.}\\
    A single frame \underline{must} be used to represent each scene of the segmented long video.
    
    \item \textbf{The same features from the offline feature extraction and online retrieval phases \underline{should} be used for the shot boundary detection algorithm.}
\end{enumerate}

\subsubsection{General Requirements}

\begin{enumerate}
    \item The MVP\footnote{Minimum Viable Product} \textbf{must} be a CLI\footnote{Command Line Interface}.
    \item The interface \textbf{must} accept command line inputs.
    \item The system \textbf{must} be able to run on a machine equipped with a terminal.
    \item The system \textbf{should} be able to run on Windows, Mac and Linux distributions.
    \item The system \textbf{should} be able to run on any desktop device equipped with a Python environment running with version 3.7 or higher.
    \item The system \textbf{should} run smoothly.
    \item The system \textbf{must} include a ``README'' file with instructions on how to use the system.
    \item The interface \textbf{should} be able to either directly record a video or to use a pre-recorded video as input.
    \item The interface \textbf{may} be a graphical interface e.g. a web page or a native desktop application.
\end{enumerate}

\subsection{Code Design Requirements}

\begin{enumerate}
    \item The system code \textbf{must} be written in Python version 3.7 or higher.
    \item The system code \textbf{must} use the OpenCV python library.
    \item The system code \textbf{must} be source controlled using Git.
    \item The system code \textbf{must} be backed up on GitHub.
    \item The system code's GitHub repository \textbf{must} be private during the development phases.
    \item The system \textbf{must} be covered with unit tests.
    \item The python code \textbf{should} follow PEP8 coding guidelines \cite{pep8}.
    \item The system \textbf{should} be covered with integration tests.
    \item The system \textbf{may} be covered with acceptance tests.
\end{enumerate}

\subsection{Data Requirements}

\begin{enumerate}
    \item The database videos \textbf{must} not violate any copyright laws.
    \item The video data length \textbf{must} range from 10 to 15 seconds.
    \item There \textbf{must} be a minimum of 20 unique videos in the database.
    \item The video database \textbf{must} easily be implemented with Python.
    \item The videos in the database \textbf{should} vary in style e.g. include movie extracts, animated pictures, cartoons, mangas, black and white movies, etc.
    
\end{enumerate}

\section{Summary}

This chapter has summarised the requirements needed to design and implement a successful system based on the problems formulated in Chapter \ref{ch:chapter1} and the Literature and Technology Survey in Chapter \ref{ch:chapter2}. Now that the requirements have been established, different potential solutions will be explored in the next chapter, including the benefits and disadvantages of each method, before one solution can be chosen and implemented.