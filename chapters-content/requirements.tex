This chapter gives a brief overview of the 

The following requirements are prioritised using the words \textit{must}, \textit{should} and \textit{may}; where \textit{must} indicates a crucial requirement to the system, \textit{should} an optimal requirement and \textit{may} a beneficial requirement. In the following requirements, the term "\textit{recorded video}" corresponds to the video that is used as input to be pattern matched to one of the videos in the database. It may either be directly recorded using a camera or it may be pre-recorded and passed as input.

\section{Functional Requirements}

\subsection{System Requirements}

\begin{enumerate}
    \item The system \textbf{must} be able to match the recorded video with one of the videos in the database with at least 80\% accuracy.
    \item The system \textbf{must} stabilise the recorded video before pattern matching it against the videos in the database.
    \item If using a classifier approach to pattern-matching, the system \textbf{must} train the model only a single time after a change in the video database e.g. a new video being added. It must not train the model every time the system is started.
    \item The system \textbf{must} be able to record videos or accept pre-recorded videos as input.
    \item The system \textbf{should} be able to find a match to the recorded video in less than 10 seconds.
    \item The system \textbf{should} crop the recorded video to ignore the surroundings of the clip being pattern matched.
    \item The system \textbf{may} start the pattern matching phase only when the video clip in the recorded video starts playing to avoid wasting computational time and accuracy on still images.
    \item The features used to pattern match the videos \textbf{may} be saved in a Gaussian Mixture Model.
    \item The system \textbf{may} use the Lucas-Kanade Optical Flow method to track movements in the videos.
    \item The system \textbf{may} use Fisher Vectors to crop the recorded video and improve the accuracy.
\end{enumerate}

\subsection{Graphical User Interface Requirements}

\begin{enumerate}
    \item The interface \textbf{must} be created using Tkinter.
    \item The interface \textbf{must} be very simple and intuitive to use.
    \item The interface \textbf{should} start with an option to directly record a video or submit a pre-recorded video as input
    \item The interface \textbf{should} display a progress bar while the recorded video is being processed, 
    \item The interface \textbf{should} display the result once the pattern matching has been completed, which could either be the matching video from the database or an error message.
\end{enumerate}

\subsection{System Configuration Requirements}

\begin{enumerate}
    \item The system \textbf{must} be a desktop application.
    \item The system \textbf{must} record the query video using a hand held device to simulate a mobile device recording.
    \item The system \textbf{should} be able to run on Windows, Mac and Linux distributions.
    \item The system \textbf{should} be able to run on any desktop device equipped with a Python environment running with version 3.5 or more.
    \item The system \textbf{should} run smoothly.
\end{enumerate}

\subsection{Code Design Requirements}

\begin{enumerate}
    \item The system code \textbf{must} be written in Python version 3.5 or higher.
    \item The system code \textbf{must} use the OpenCV python library.
    \item The system code \textbf{must} be source controlled using Git.
    \item The system code \textbf{must} be backed up on GitHub.
    \item The system code's GitHub repository \textbf{must} be private during the development phases.
    \item The system \textbf{must} be covered with unit tests.
    \item The python code \textbf{should} be follow PEP8 coding guidelines \cite{pep8}.
    \item The system \textbf{should} be covered with integration tests.
    \item The system \textbf{may} be covered with acceptance tests.
\end{enumerate}

\subsection{Data Requirements}

\begin{enumerate}
    \item The video data \textbf{must} not violate any copyright laws.
    \item The video data \textbf{must} range from 5 to 15 seconds.
    \item There \textbf{must} be a minimum of 10 unique videos in the database.
    \item The video database \textbf{must} easily be implemented with Python.
    \item The video data \textbf{should} be stored in a PostGreSQL database.
    \item The videos in the database \textbf{should} vary in style e.g. include movies extracts, animated pictures, cartoons, mangas, black and white movies, etc.
    
\end{enumerate}

\section{Non-Functional Requirements}

\begin{enumerate}
    \item The system \textbf{must} have a name.
    \item The interface \textbf{should} be easy to read, using large fonts.
    \item The interface \textbf{should} be easy to navigate, using icons and navigation elements.
    \item The system \textbf{may} not crash or exit unexpectedly.
\end{enumerate}