This is the chapter in which you review the major achievements in the
light of your original objectives, critique the process, critique your
own learning and identify possible future work.

Discussion:
\begin{itemize}
    \item state what could be improved with time
    \begin{itemize}
        \item region-based histograms, separate frame in 5 regions and generate a histogram for each region rather than having a global histogram to describe the frame (https://www.pyimagesearch.com/2014/12/01/complete-guide-building-image-search-engine-python-opencv/)
        \item improve ROI selection to take 4 points rather than 2 (will increase accuracy for query videos that are at an angle, not filming the screen from a straight point of view)
        \item automatic ROI detection (detect screen edges with edge and corner detector algorithms)
        \item more features to complement the colour-based features (e.g. motion and shape features) and improve the accuracy
        \item make use of sound to further improve accuracy
        \item GUI e.g. Tkinter application or simple HTML webpage
    \end{itemize} 
    \item mention that with improvements, could have worked with small database of movies rather than short videos
    \item limitations of working with large database of movies: 
    \begin{itemize}
        \item too much data to process, could perhaps be pre-processed in advance
        \item copyright issues of having all these movies
    \end{itemize}
\end{itemize}