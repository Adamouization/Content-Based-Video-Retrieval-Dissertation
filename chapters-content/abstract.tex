This project presents the design concepts and implementation steps of a content-based retrieval system for videos. Nowadays, unstructured data grows at exponential rates, and content-based retrieval systems can help improve the problem. Most of this unorganised data originating from social networks exists in the form of videos, which is why the task of retrieving videos from large databases is an important one.\\

The project was originally inspired by the famous music-matching mobile application Shazam, with the aim to create a similar system for matching movies rather than music in order to address the previously mentioned problem. However, an application of this scope has natural limitations due to the colossal size that a database of movies would occupy and the legal issues of employing such a database of movies for an application. Therefore, the aim of this dissertation is to create a prototype version of the system and later explore potential improvements to overcome these limitations.\\

Ultimately, a functional system was built by combining multiple methods into one pipeline and tested with a database of 50 short videos along with various videos recorded through mobile phones, resulting in correct matches reaching accuracies of 93\%. To increase the challenge and realism of the tests, the recorded queries attempted to imitate what user-recorded videos would look like by replicating videos of poor quality with shaking motions and inadequate framing, which the system managed to cope with to an extent. The results were then compared to an online experiment conducted to establish ground truth, which required participants to play the role of the system and match a query video to a database video. To complete the pipeline, a feature-length movie was used to test how it could be condensed into one image per shot in order to be used by the matching algorithm in the future.\\

The code developed for this dissertation can be found online at the following URL: \url{https://github.com/Adamouization/Content-Based-Video-Retrieval-Code}.